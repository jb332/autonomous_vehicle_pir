\sectionnn{Conclusion}

Designing an autonomous vehicle is challenging in many ways, requiring several advanced technologies to mimic what a driver does. Nevertheless, the increase of computational power and the boosting IA development make AV a trend in the domain of research. It makes us believe that this technology will soon reach a mature state of development and eventually become reliable enough to be commercialised someday. Such an outcome could revolutionize the transport in reducing costs, traffic congestion and above all the accidents rate.
\smallskip

As the world continues to move progressively toward a transportation system driven by AV, what AV could bring up to the society is significant. 
AV has potential to transfer urban mobility by offering the opportunity for an efficient, safe, accessible and affordable transportation. They promise not only an alternative system of mobility but also a new approach to the urban lifestyle. Even though these benefits are far from guaranteed.
\smallskip

The existing technologies today allows us to detect obstacles and road condition by reconstructing the environment from pictures (by using cameras) or from points (by using LiDAR). However, the body gesture of the pedestrian is not taken in consideration of the calculation. Pedestrians are "seen" only as an obstacle under the eyes of the sensors. 

\smallskip

%platform connecte(Remy) - connectivity(denis)
The communication between connected cars is a not a piece of cake. The communication of AV is based on different layer : communication vehicle to vehicle, communication vehicle to road infrastructure, communication vehicle to smart device or communication vehicle to Internet. Communication of vehicle to infrastructure allows control stations to manage driving areas and improves GPS application or even to detect violation such as red light violation.
\smallskip

We studied the OM2M framework in particular. It is based on the oneM2M standard and makes possible to precise the semantic of the information exchanged between devices. It offers a graphical interface to make it easier to use. Therefore, it is the framework we are going to use for our project.\smallskip

The INSA vehicle possesses LiDAR and camera that could be used to get information about the campus. For instance, it allows us to know the number of available parking slots and their positions, or the number of people queuing for the restaurant, or even analyses paths taken by students to get to class.